\documentclass[10pt]{article}
\usepackage[brazil]{babel}
\usepackage[utf8]{inputenc}
\usepackage[T1]{fontenc}
\usepackage{amsmath}

%Qual a diferença entre as Leis de Amdahl e Gustafson?
    
\title{Diferença entre as Leis de Amdahl e Gustafson\\OPRP}
    
\author{Aluno: \textbf{Gustavo José Neves da Silva}}{


\begin{document}
    
    \date{}
    
    \maketitle
    
    \section*{Lei de Amdahl}
    
    \begin{align*}
    	\text{Aceleração}&=\frac{(s + p)}{(s + p/n)} \\
    					 &=\frac{1}{(s+p/n)}    	
    \end{align*}
    
    Tal que:
    \begin{align*}
    	n &= \text{número de processadores} \\
    	p &= \text{quantia de tempo gasta em partes paralelizáveis} \\
    	s &= \text{quantia de tempo gasta na parte sequencial} \\
    	s+p &= 1, \text{porque é o total de tempo necessário para executar o código}
    \end{align*}

	Considera que o tamanho do problema é fixo independentemente do número de processadores disponíveis.
    
    \section*{Lei de Gustafson}
    
    \begin{align*}
    	\text{Aceleração escalável}&=\frac{(s + p \times N)}{(s + p)} \\
    					 &= s + p \times N \\
    					 &= N + (1 - N) \times s    	
    \end{align*}
    
    Tal que:
    \begin{align*}
    	N &= \text{número de processadores} \\
    	p &= \text{quantia de tempo gasta em partes paralelizáveis} \\
    	s &= \text{quantia de tempo gasta na parte sequencial}
    \end{align*}

	Considera que o tamanho do problema escala em conjunto com o número de processadores disponíveis. 
    

\end{document}
\documentclass[10pt]{article}
\usepackage[brazil]{babel}
\usepackage[utf8]{inputenc}
\usepackage[T1]{fontenc}

%Após efetuar a leitura, elabore uma resenha (no máximo uma página) sobre o conteúdo do artigo.

%  No texto, deixe claro:
%     contribuição do trabalho
%     pontos positivos
%     pontos negativos
%     e sua opinião sobre o tema (com argumentos).

    
\title{Resenha - OPRP}
    
\author{Aluno: \textbf{Gustavo José Neves da Silva}}{


\begin{document}
    
    \date{}
    
    \maketitle

    \section{Contribuição do trabalho}
    O artigo resenhado, contribui de forma primorosa para o entendimento dos conceitos relativos a Grid Computing e Cloud Computing, pois os expôs de forma clara e abrangente.
    Para tanto realizou comparações entre os paradigmas tendo em vista: a arquitetura, o modelo de segurança, o modelo de negócios, o modelo de programação, a virtualização, o modelo de dados, o modelo de computação, a forma que é provido e as aplicações.

    \section{Pontos positivos}
    \begin{itemize}
        \item Clareza ao explicar os conceitos, utilizando de analogias e de exemplos para maior compreensão
        \item Abrangência do conteúdo: o artigo consegui abringir todo(tendo como parâmetro a época de publicação) o conteúdo 
    \end{itemize}
    \section{Pontos negativos}
    \begin{itemize}
        \item Checklist de três itens - Ian Foster
            \begin{enumerate}
                \item A coordenação dos recursos não pode estar sob um controle centralizado
                \item Utilizar interfaces e protocolos padrão, abertos e de propósito geral
                \item Entregar qualidades não triviais dos serviços
            \end{enumerate}
            Apenas o item 3 é válido para Cloud Computing, o 1 e 2 ainda não estão claros, ou seja, dependem da empresa provedora do serviço.

        \item  Atualmente a segurança de Clouds possui maior eficiência comparada a oferecida na época da publicação do artigo.
    \end{itemize}

    \section{Opinião sobre o tema}
    Cloud computing é um tipo de grid, com a diferença de ser escalável, de oferecer ao usuário uma abstração dos recursos e de um dos limitantes ser a quantia monetária disponível para investimento.

\end{document}